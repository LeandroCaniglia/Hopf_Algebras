\chapter{Hopf Algebras}

\section{Introduction}

In this section $\kappa$-coalgebras are not necessarily commutative or cocommutative. Examples of Hopf algebras are provided in Chapter~\ref{ch:representable} (\ref{xmpls:k-group-schemes} and \ref{exr:ring-group-hopf-algebra}).

\begin{defn}
    Let $H$ be a $\kappa$-Hopf algebra with counit map $\varepsilon: H \to\kappa$. A \textsl{left integral} of $H$ is an element $y\in H$ that satisfies
    \begin{equation}\label{eq:left-integral}
        xy=\varepsilon(x)y
    \end{equation}
    for all $x\in H$. Similarly, $y$ is a \textsl{right integral} if it satisfies
    \begin{equation}\label{eq:right-integral}
        yx = \varepsilon(x)y
    \end{equation}
    We denote the collection of left integrals of $H$ by $\int_H^l$ and the collection of right integrals by $\int_H^r$. If $H$ is commutative left and right integral elements all called \textsl{integral} and their collection denoted by $\int_H$.
\end{defn}

\begin{xmpl}
    Let $G$ be a finite abelian group and $\kappa[G]$ the group ring with coefficients in $\kappa$. Suppose that $y\in\kappa[G]$ is integral in the $\kappa$-Hopf algebra structure of $\kappa[G]$ [cf.~Exercise~\ref{exr:ring-group-hopf-algebra}]. Put
    $$
        y = \sum_{\sigma\in G}b_\sigma\sigma.
    $$
    
    Then, from \eqref{eq:left-integral} applied to $x=\gamma$, we get
    $$
        \sum_{\sigma\in G}b_\sigma\gamma\sigma
            = \sum_{\sigma\in G}b_\sigma\sigma.
    $$
    It follows that $b_{\gamma\sigma}=b_\sigma$ for all $\gamma$ and all $\sigma$. In particular, $b_\sigma=b_{\sigma^{-1}\sigma}=b_{1_G}$, where $1_G$ is the identity of $G$. Hence, $y=b_{1_G}\sum_{\sigma\in G}\sigma$, i.e., $y$ is in the ideal generated by $\sum_{\sigma\in G}\sigma$.
    
    To show the converse, first observe that
    $$
        xy=\varepsilon(x)y\text{ and }zy=\varepsilon(z)y
            \implies(x+cz)y = \varepsilon(x+cz)y.
    $$
    Thus, to see that $y\in\int_H$ it is enough to verify that $\gamma y=y$ for all $\gamma\in G$. But this is clear in the case we are analyzing:
    $$
        \gamma y = \gamma\Big(\sum_{\sigma\in G}\sigma\Big)
            = \sum_{\sigma\in G}\gamma\sigma = \sum_{\zeta\in G}\zeta = y.
    $$
    In consequence,
    $$
        \int_H = \gen{\sum_{\sigma\in G}\sigma}.
    $$
\end{xmpl}

\begin{prop}
    Let $H$ be a $\kappa$-Hopf algebra. Then $\int_H^l$ and $\int_H^r$ are ideals of~$H$.
\end{prop}

\begin{proof} It is enough to consider $\int_H^l$. Clearly $0\in\int_H^l$. Moreover $\int_H^l$ is closed under sums:
    \begin{align*}
        xy = \varepsilon(x)y\text{ a}&\text{nd }xy'=\varepsilon(x)y'\\
            &\!\Downarrow\\
            x(y+y') = xy + xy' &= \varepsilon(x)y + \varepsilon(x)y'
                = \varepsilon(x)(y+y').
    \end{align*}
    Finally, for $x,z\in H$ and $y\in\int_H$, we have
    $$
        x(zy) = (xz)y = \varepsilon(xz)y = \varepsilon(x)\varepsilon(z)y
            = \varepsilon(x)(zy).
    $$
\end{proof}


\begin{defns}${}$
    \begin{enumerate}[-]
        \item Let $A$ and $B$ be $\kappa$-coalgebras. A $\kappa$-linear map $\phi\colon A\to B$ is a \textsl{morphism of $\kappa$-coalgebras} it satisfies
        \begin{enumerate}[\rm i)]
            \item $(\phi\otimes_\kappa\phi)\circ\Delta_A=\Delta_B\circ\phi$.
            \item $\varepsilon_A=\varepsilon_B\circ\phi$.
        \end{enumerate}
    
        \item If $A$ and $B$ are $\kappa$-bialgebras, $\phi$ is a \textsl{morphism of $\kappa$-bialgebras} if it is a morphism of algebras and a morphism of coalgebras.
    
        \item {\rm[cf.~Theorem~\ref{thm:k-group-scheme-equivalence}]} If $A$ and $B$ be $\kappa$-Hopf algebras, $\phi$ is a \textsl{morphism of $\kappa$-Hopf algebras} if it is a morphism of $\kappa$-bialgebras and satisfies
        \begin{enumerate}
            \item[iii)] $\phi\circ\sigma_A=\sigma_B\circ\phi$.
        \end{enumerate}
    \end{enumerate}
\end{defns}

\begin{prop}
    Let $A$ be a $\kappa$-bialgebra with multiplication $\mu_A\colon A\otimes_\kappa A\to A$. Then $\mu_A$ is a morphism of $\kappa$-bialgebras. Moreover, if $A$ is a commutative $\kappa$-Hopf algebra, then $\mu_A$ is a morphism of $\kappa$-Hopf algebras.
\end{prop}

\begin{proof} Take $a,b\in A$.
    \begin{enumerate}[\rm i)]
        \item With the usual notations we have
        %\small
        \begin{align*}
            (\mu_A\otimes\mu_A)
                    \circ\Delta_{A\otimes A}(a\otimes b)
                &= (\mu_A\otimes\mu_A)\Big(\sum_{i=1}^n\sum_{j=1}^m
                    a_{1i}\otimes b_{1j}\otimes a_{2i}\otimes b_{2j}\Big)
                        \\[-0.1in]
                &= \sum_{i=1}^n\sum_{j=1}^m
                    a_{1i}b_{1j}\otimes a_{2i}b_{2j}\\
                &= \sum_{i=1}^n\sum_{j=1}^m
                    (a_{1i}\otimes a_{2i})(b_{1j}\otimes b_{2j}).\\
                &= \Delta_A(a)\Delta_A(b)\\
                &= \Delta_A(ab)\\
                &= \Delta_A\circ\mu_A(a\otimes b).
        \end{align*}
        \normalsize

        \item 
        %\small
        \begin{align*}
            \varepsilon_{A\otimes A}(a\otimes b)
                &= \varepsilon_A(a)\varepsilon_A(b)\\
                &= \varepsilon_A(ab)\\
                &= \varepsilon_A\circ\mu_A(a\otimes b).
        \end{align*}
        \normalsize

        \item
        %\small
        \begin{align*}
            \mu_A\circ\sigma_{A\otimes A}(a\otimes b)
                &= \mu_A(\sigma_A\otimes\sigma_A(a\otimes b))\\
                &= \mu_A(\sigma_A(a)\otimes\sigma_A(b))\\
                &= \sigma_A(a)\sigma_A(b)\\
                &= \sigma_A(ba)
                    &&\text{; Lem.~\ref{lem:*-coinverse-ok}}\\
                &= \sigma_A(ab)
                    &&\text{; $A$ commut.}\\
                &= \sigma_A\circ\mu_A(a\otimes b).
        \end{align*}
        \normalsize
    \end{enumerate}
\end{proof}

\section{Dual and Finite Dual}

Let $\kappa$ be a commutative ring. Recall that $\Hom_\kappa(-,\kappa)$ is a contravariant functor that maps every $\kappa$-module map $\phi\colon V\to W$ to $\phi^*\colon W^*\to V^*$, where $\phi^*$ is the \textsl{transpose} of~$\phi$.

\begin{lem}\label{lem:dual-kappa-algebra}
    If\/ $C$ is a\/ $\kappa$-coalgebra, then\/ $C^*$ is an\/ $\kappa$-algebra with multiplication\/ $\Delta\!^*$ and unit\/ $\varepsilon^*$.
\end{lem}

\begin{proof}
    Put $\iota\colon C^*\otimes_\kappa C^*\to (C\otimes_\kappa C)^*$ and define $\mu\colon C^*\otimes_\kappa C^*\to C^*$ as $\mu = \Delta\!^*\circ\iota$. Then $\mu$ is the multiplication operation in $C^*$:
    $$
        \mu(f\otimes g) = \iota(f\otimes g)\circ\Delta.
    $$
    Thus, given $f,g\in C^*$, if we denote $\mu(f\otimes g)$, as usual, by $fg$, we get
    $$
        fg(x) = \sum_{i=1}^nf(x_{1i})g(x_{2i})\quad\text{ where }
            \Delta(x) = \sum_{i=1}^nx_{1i}\otimes x_{2i}.
    $$
    In other words, the multiplication in $C^*$ is the convolution operation. Note that, for $k\in\kappa$, after identifying the homotecy $\eta_k$ with $k$, we obtain
    $$
        \varepsilon^*(k)(c) = \eta_k\circ\varepsilon(c)
            = k\varepsilon(c),
    $$
    i.e., $\varepsilon^*(k)=k\varepsilon$. In particular, $\varepsilon^*(1)=\varepsilon$, which shows that $\varepsilon^*\colon\kappa\to C^*$ is the inclusion $\iota_{C^*}\colon\kappa\to C^*$ or, equivalently, $\varepsilon=1_{C^*}$.

    The convolution is associative by Lemma~\ref{lem:*-associativity} and that $\varepsilon^*$ is the unit by Lemma~\ref{lem:*-identity}. It remains to be seen that $f(g+h)=fg+fh$, but
    \begin{align*}
        f(g+h)(x) &= \sum_{i=1}^nf(x_{1i})(g(x_{2i})+h(x_{2i}))\\
            &= \sum_{i=1}^nf(x_{i1})g(x_{2i}) + f(x_{1i})h(x_{2i})\\
            &= (fg+fh)(x).
    \end{align*}
\end{proof}

\begin{rem}\label{rem:algebra-to-coalgebra-issue}
    As we have seen, if we start with a $\kappa$-coalgebra $C$ we obtain a $\kappa$-algebra $C^*$. However, we cannot always go from a $\kappa$-algebra $A$ to a $\kappa$-coalgebra $A^*$ using $\mu_A^*$ as $\Delta$ because $\mu_A^*\colon A^*\to(A\otimes_\kappa A)^*$, whose image may fail to belong in $A^*\otimes A^*$. See however Corollary~\ref{cor:algebra-to-coalgebra}
\end{rem}

\begin{defn}
    Let $C$ be a $\kappa$-coalgebra. A $\kappa$-module $M$ is a \textsl{right $C$-comodule} if there exists a $\kappa$-linear map $\delta\colon M\to M\otimes_\kappa C$, such that the diagrams
    $$
        \begin{tikzcd}
            M
                    \arrow[d,"\delta"']
                    \arrow[r,"\delta"]
                &M\otimes_\kappa C
                    \arrow[d,"\id_M\otimes\Delta"]
                &&M
                    \arrow[r,"\delta"]
                    \arrow[rd,"\otimes1"']
                &M\otimes_\kappa C
                    \arrow[d,"\id_M\otimes\varepsilon"]\\
            M\otimes_\kappa C
                    \arrow[r,"\delta\otimes\id_C"']
                &M\otimes_\kappa C\otimes_\kappa C
                &&&M\otimes_\kappa\kappa
        \end{tikzcd}
    $$
    commute.
\end{defn}

\begin{rem}
    There is an analogous definition of \textsl{left $C$-module} where the map $\delta$ goes from $M$ to $C\otimes_\kappa M$.
\end{rem}

\begin{ntn}
    If\/ $M$ is right\/ $C$-comodule, given\/ $x\in M$ we will write
    $$
        \delta(x) = \sum_{i=1}^nx_{0i}\otimes x^{1i},
    $$
    where\/ $x_{0i}\in M$ and\/ $x^{1i}\in C$ for\/ $1\le i\le n$. In the case of a left\/ $C$-comodule we will write
    $$
        \delta(x) = \sum_{i=1}^nx^{1i}\otimes x_{0i}.
    $$
\end{ntn}

\begin{rem}\label{rem:right-comodule}
    With the preceding notations, we have
    \begin{align*}
        (\delta\otimes\id_C)\circ\delta(x)
            &= \sum_{i=1}^n\delta(x_{0i})\otimes x^{1i}
            = \sum_{i=1}^n\sum_{j=1}^{n_i}
                (x_{0i})_{0j}\otimes(x_{0i})^{1j}\otimes x^{1i}\\
        (\id_M\otimes\Delta)\circ\delta(x)
            &= \sum_{i=1}^nx_{0i}\otimes\Delta(x^{1i})
            = \sum_{i=1}^n\sum_{h=1}^{m_i}
                x_{0i}\otimes(x^{1i})_{1h}\otimes(x^{1i})_{2h}.
    \end{align*}
\end{rem}


\begin{prop}\label{prop:c-comodule-is-left-dual-comodule}
    Let $C$ be a $\kappa$-coalgebra. Then, every right\/ $C$-comodulue is a left\/ $C^*$-module.
\end{prop}

\begin{proof}
    Suppose that $M$ is a right $C$-comodule. Define
    \begin{align*}
        \cdot\colon C^*\times M&\to M\\
            (f,x) &\mapsto f\cdot x=\sum_{i=1}^nf(x^{1i})x_{0i},
    \end{align*}
    i.e.,
    $$
        f\cdot x=\mu\circ(\id_M\otimes f)\circ\delta(x),
    $$
    which ---in addition--- shows that the definition of $f\cdot x$ doesn't depend on the representation of $\delta(x)$ we happen to choose. Note also that
    $$
        \delta\Big(\sum_{i=1}^nf(x^{1i})x_{0i}\Big)
            = \sum_{i=1}^nf(x^{1i})\delta(x_{0i})
            = \sum_{i=1}^n\sum_{j=1}^{n_i}
                f(x^{1i})((x_{0i})_{0j}\otimes(x_{0i})^{1j}).
    $$
    Therefore,
    \begin{align*}
        g\cdot(f\cdot x) &= \sum_{i=1}^n\sum_{j=1}^{n_i}
                f(x^{1i})g((x_{0i})^{1j})(x_{0i})_{0j}\\
            &= \mu\circ(\id_M\otimes g\otimes f)\Big(
                \sum_{i=1}^n\sum_{j=1}^{n_i}
                (x_{0i})_{0j}\otimes(x_{0i})^{1j}\otimes x^{1j}\Big)\\
        gf\cdot x &= \sum_{i=1}^ngf(x^{1i})x_{0i}\\
            &= \sum_{i=1}^n\sum_{h=1}^{m_i}
                g((x^{1i})_{1h})f((x^{1i})_{2h})x_{0i}\\
            &= \mu\circ(\id_M\otimes g\otimes f)\Big(
                 \sum_{i=1}^n\sum_{h=1}^{m_i}
                 x_{0i}\otimes(x^{1i})_{1h}\otimes(x^{1i})_{2h}\Big)
    \end{align*}
    and Remark~\ref{rem:right-comodule} shows that $g\cdot(f\cdot x) = gf\cdot x$.

    To see that the identity of $C^*$ acts trivially on $M$ observe that, using the second diagram of the definition of comodule, we obtain
    \begin{align*}
        \varepsilon\cdot x &= \sum_{i=1}^n\varepsilon(x^{1i})x_{0i}\\
            &= \mu\circ(\id_M\otimes\varepsilon)\circ\delta(x)\\
            &= \mu(x\otimes1)\\
            &= x.
    \end{align*}
    The remaining properties, namely $(f+g)\cdot x=f\cdot x + g\cdot x$ and $f\cdot (x+y) = f\cdot x + f\cdot y$ follow directly from the $\kappa$-linearity of the quantities involved.
\end{proof}

\begin{defn}
    Suppose that $\kappa$ is field. Given a $\kappa$-algebra $A$ the \textsl{finite dual} of $A$ is
    $$
        A^\circ = \set{f\in A^*\mid\dim(A/\ker f)<\infty}.
    $$
\end{defn}

\begin{rem}
    It's worth mentioning that $A^\circ$ is a vector subspace of $A$. Indeed, given $f_1,f_2\in A^\circ$ and $k\in\kappa$, we have
    $$
        (\ker f_1)\cap(\ker f_2)\subseteq\ker(f_1+kf_2).
    $$
    Consider the following commutative diagram
    $$
    \begin{tikzcd}
        A
                \arrow[d,"\varphi"']
                \arrow[r,"{(\varphi_1,\varphi_2)}"]
            &A/\ker f_1\oplus A/\ker f_2\\
        A/(\ker f_1)\cap(\ker f_2)
            \arrow[ru,"\psi"',dashed]
    \end{tikzcd}
    $$
    Clearly $\psi$ is an injection. In particular $\dim A/(\ker f_1)\cap(\ker f_2)<\infty$ and therefore $\dim A/\ker(f_1+kf_2)<\infty$.
\end{rem}

\begin{defn}\label{defn:b-module-algebra}
    Let $B$ a $\kappa$-bialgebra and $A$ a $\kappa$-algebra that is also a $B$-module. We say that $A$ is a \textsl{left $B$-module algebra} if the module action of $B$ on $A$ satisfies
    $$
        b\cdot aa' = \sum_{i=1}^n(b_{1i}\cdot a)(b_{2i}\cdot a')
        \quad\text{and}\quad
        b\cdot1_A=\varepsilon_B(b)1_A.
    $$
    If $\lambda\colon B\otimes A\to A$ denotes the action, the first condition can be written as
    $$
        \lambda\circ(\id_B\otimes\mu_A)
            =\mu_A\circ(\lambda\otimes\lambda)
                \circ(\id_B\otimes t\otimes\id_A)
                \circ(\Delta_B\otimes\id_A\otimes\id_A),
    $$
    which translates into
    $$
        \begin{tikzcd}[column sep=huge]
            B\otimes_\kappa A\otimes_\kappa A
                    \arrow[rr,"\lambda\circ(\id_B\otimes_\kappa\mu_A)"]
                    \arrow[d,"\Delta_B\otimes\id_A\otimes\id_A"']
                &&A\\
            B\otimes_\kappa B\otimes_\kappa A\otimes_\kappa A
                    \arrow[r,"\id_B\otimes t\otimes\id_A"']
                &B\otimes_\kappa A\otimes_\kappa B\otimes_\kappa A
                    \arrow[r,"\lambda\otimes\lambda"']
                &A\otimes A
                    \arrow[u,"\mu_A"']
        \end{tikzcd}
    $$
\end{defn}


\begin{defn}
    Let\/ $A$ be a\/ $\kappa$-algebra. Given\/ $f\in A^*$ and $a\in A$ the \textsl{left translate} of $f$ by $a$ is
    $$
        (a\rharpoon f)(b) = f(ba),
    $$
    for $b\in A$. Similarly, the \textsl{right translate} is defined as
    $$
        (f\lharpoon a)(b) = f(ab).
    $$
\end{defn}

\begin{rem}
    Note that both $a\rharpoon f$ and $f\lharpoon a$ belong in $A^*$:
    \begin{align*}
        (a\rharpoon f)(b+kb') &= f((b+kb')a)\\
            &= f(ba)+kf(b'a)\\
            &= (a\rharpoon f)(b) + k(a\rharpoon f)(b').
    \end{align*}
    and
    \begin{align*}
        (f\lharpoon a)(b+kb') &= f(a(b+kb'))\\
            &= f(ab) + kf(ab')\\
            &= (f\lharpoon a)(b) + k(f\lharpoon a)(b').
    \end{align*}
\end{rem}

\begin{prop}
    Let $B$ be a $\kappa$-bialgebra. The left translate is an action of $B$ on $B^*$ that endows on $B^*$ a structure of $B$-module algebra.
\end{prop}

\begin{proof}
    First recall that $B^*$ is a $\kappa$-algebra with the convolution multiplication [cf.~Lemma~\ref{lem:dual-kappa-algebra}]. On the other hand, given $b\in B$ and $f,f'\in B^*$, we have
    \begin{align*}
        (b\rharpoon ff')(x) &= ff'(xb)\\
            &= \sum_{i=1}^n\sum_{j=1}^mf(x_{1i}b_{1j})f'(x_{2i}b_{2j})\\
            &= \sum_{i=1}^n\sum_{j=1}^m(b_{1j}\rharpoon f)(x_{1i})
                (b_{2j}\rharpoon f')(x_{2i})\\
            &= \sum_{j=1}^m(b_{1j}\rharpoon f)(b_{2j}\rharpoon f')(x),
    \end{align*}
    as required by the first condition of Definition~\ref{defn:b-module-algebra}. For the second, note that
    \begin{align*}
        (1_B\rharpoon\varepsilon_B)(x) &= \varepsilon(x1_B)\\
            &= \varepsilon_B(x),
    \end{align*}
    as desired, since $\varepsilon_B=1_{B^*}$ as shown by Lemma~\ref{lem:dual-kappa-algebra}.
\end{proof}

\begin{rem}\label{rem:A-rharpoon-f-lharpoon A}
    Given a $\kappa$-algebra $A$ and $f\in A^*$ the set
    $$
        A\rharpoon f = \set{a\rharpoon f\mid a\in A}
    $$
    is a $\kappa$-submodule of $A$. Indeed,
    \begin{align*}
        ((a + ka')\rharpoon f)(b) &= f(ba) + kf(ba')\\
            &= (a\rharpoon f)(b) + k(a'\rharpoon f)(b)\\
            &= ((a\rharpoon f) + k(a'\rharpoon f))(b).
    \end{align*}
    Similarly for
    $$
        f\lharpoon A = \set{f\lharpoon a\mid a\in A}.
    $$
    Note also that
    \begin{align*}
        ((a\rharpoon f)\lharpoon a')(b) &= (a\rharpoon f)(a'b)\\
            &= f(a'ba).\\
        (a\rharpoon(f\lharpoon a'))(b) &= (f\lharpoon a')(ba)\\
            &= f(a'ba).
    \end{align*}
    In consequence, we can represent the common value of these linear forms by $a\rharpoon f\lharpoon a'$. In particular, we have the $\kappa$-submodule of $A$
    $$
        A\rharpoon f\lharpoon A
            = \set{a\rharpoon f\lharpoon a'\mid a,a'\in A}.
    $$
\end{rem}

\begin{lem}
    Let $A$ be a $\kappa$-algebra. Given $f\in A^*$ put
    $$
        I = \bigcap_{a\in A}{\ker (a\rharpoon f)}.
    $$
    Then, $I$ is a right ideal of $A$ and there is $\kappa$-linear monomorphism
    $$
        A/I\to(A\rharpoon f)^*.
    $$
\end{lem}

\begin{proof}
    Define
    \begin{align*}
        \ev\colon A&\to(A\rharpoon f)^*\\
        b&\mapsto\ev_b,
    \end{align*}
    where $\ev_b(a\rharpoon f)=(a\rharpoon f)(b)$, for all $a\in A$. In particular,
    $$
        b\in\ker(\ev) \iff b\in\ker(a\rharpoon f)(b)=0 \text{ for all }a\in A \iff b\in I.
    $$
    Thus, $I=\ker(\ev)$ and $\ev$ induces a $\kappa$-linear injection from $A/I$ to $(A\rharpoon f)^*$.
\end{proof}

\begin{thm}
    Suppose that\/ $\kappa$ is a field and let\/ $A$ be a\/ $\kappa$-algebra. Given\/ $f\in A^*$, the following are equivalent:
    \begin{enumerate}[\rm 1)]
        \item $f$ vanishes on a right ideal of\/ $A$ of finite codimension
        \item $f$ vanishes on a left ideal of\/ $A$ of finite codimension
        \item $f$ vanishes on an ideal of\/ $A$ of finite codimension
        \item $\dim(A\rharpoon f)<\infty$
        \item $\dim(f\lharpoon A)<\infty$
        %\item $\dim(A\rharpoon f\lharpoon A)<\infty$
        \item\label{item:last} $\mu^*(f)\in A^*\otimes A^*$
    \end{enumerate}
    Consequently, $f \in A^{\circ}$ if any of conditions\/ {\rm 1)} through {\rm\ref{item:last}} does hold.
\end{thm}

\begin{proof}${}$
    \begin{enumerate}
        \item[1)]$\Rightarrow$ 4) Let $I\subseteq\ker f$ be a right ideal of finite codimension. If $a\in A$, then $(a\rharpoon f)(I)=f(Ia)\subseteq f(I)=0$. Thus, $A\rharpoon f\subseteq(A/I)^*$, which has finite dimension.

        \item[4)]$\Rightarrow$ 1) According to the preceding lemma, $A/I$ has finite dimension. Moreover, since $f=1\rharpoon f$, $I\subseteq \ker f$, i.e., $f$ vanishes on the right ideal~$I$.

        \item[2)]$\Leftrightarrow$ 5) Similar to 1)~$\Leftrightarrow$~4).

        \item[3)]$\Rightarrow$ 1) is trivial.

        \item[3)]$\Rightarrow$ 2) is trivial.

        \item[2)]$\Rightarrow$ 3) Let $J$ be a left ideal of $A$ such that $\dim A/J<\infty$ and $J\subseteq\ker f$. Define
        \begin{align*}
            \phi\colon A&\to\End_\kappa(A/J)\\
                a&\mapsto (\widebar b\mapsto\widebar{ab}),
        \end{align*}
        which is well-defined precisely because $J$ is a left ideal. Since $A/J$ is finite dimensional, $\End_\kappa(A/J)$ and therefore $A/\ker\phi$ are finite dimensional too.

        If $a,a'\in A$, we have
        $$
            \phi(aa')(\widebar b) = \widebar{aa'b}
                = \phi(a)\circ\phi(a')(\widebar b),
        $$
        i.e., $\phi(aa')=\phi(a)\circ\phi(a')$ (in other words, $\phi$ is a morphism of $\kappa$-algebras). In particular, $\ker\phi$ is a (two-sided) ideal of $A$. Take $c\in\ker\phi$. Then,
        $$
            0 = \phi(c)(\widebar1) = \widebar{c}, 
        $$
        hence, $c\in J$. In particular, $f$ vanishes on $\ker\phi$.

        \item[1)]$\Rightarrow$ 3) This is similar to 2)~$\Rightarrow$~3) with $\phi(a)$ defined as $\widebar b\mapsto\widebar{ba}$.

        \item[4)]$\Rightarrow$ \ref{item:last} Let $\set{a_1\rharpoon f,\dots,a_n\rharpoon f}$ be a basis of $A\rharpoon f$. Let $\set{h_1,\dots,h_n}$ denote the dual basis. Given $a,b\in A$, put $g_i(a) = h_i(a\rharpoon f)$, which is an element of $A^*$. Then,
        \begin{align*}
            \mu^*(f)(b\otimes a) &= f(ba)\\
                &= (a\rharpoon f)(b)\\
                &= \sum_{i=1}^ng_i(a)(a_i\rharpoon f)(b)\\
                &= \sum_{i=1}^n(a_i\rharpoon f)\otimes g_i(b\otimes a).
        \end{align*}
        Thus, $\mu^*(f)= \sum_{i=1}^n(a_i\rharpoon f)\otimes g_i\in A^*\otimes A^*$.

        \item[\ref{item:last}]$\Rightarrow$ 4) By hypothesis we can write
        $$
            \mu^*(f) = \sum_{i=1}^nh_i\otimes g_i.
        $$
        Therefore
        \begin{align*}
            (a\rharpoon f)(b) &= \mu^*(f)(b\otimes a)\\
                &= \sum_{i=1}^ng_i(a)h_i(b),
        \end{align*}
        i.e., $a\rharpoon f$ belongs in the subspace generated by $\set{h_1,\dots,h_n}$, which is finite dimensional.
    \end{enumerate}
\end{proof}

\begin{cor}
    Suppose that $\kappa$ is a field and let $A$ be a $\kappa$-algebra. Then $\mu^*(A^\circ)\subseteq A^\circ\otimes_\kappa A^\circ$.
\end{cor}

\begin{proof}
    Take $f\in A^\circ$. Then $A\rharpoon f$ has a basis $\mathcal B=\set{a_1\rharpoon f,\dots,a_n\rharpoon f}$. Let $\set{h_1,\dots,h_n}$ be the dual basis of $\mathcal B$. As shown in Remark~\ref{rem:A-rharpoon-f-lharpoon A}, every linear relation involving $\set{a_1,\dots,a_n}$ translates into the same relation among the elements of $\mathcal B$. In consequence, $\set{a_1,\dots,a_n}$ is linearly independent. Let $\set{g_1,\dots,g_n}$ be its dual basis. Given $a\in A$ we can have
    \begin{align*}
        a &= g_1(a)a_1+\cdots+g_n(a)a_n\\
        a\rharpoon f &= g_1(a)(a_1\rharpoon f)
                +\cdots+g_n(a)(a_n\rharpoon f)\\
        a\rharpoon f &= h_1(a\rharpoon f)(a_1\rharpoon f)
                +\cdots+h_n(a\rharpoon f)(a_n\rharpoon f).
    \end{align*}
    Hence, $h_i(a\rharpoon f) = g_i(a)$. In particular $h_i(a_j\rharpoon f)=\delta_{ij}$.

    On the other hand, according to the theorem, we can write
    \begin{equation}\label{eq:f-mu}
        \mu^*(f) = \sum_{i=1}^n(a_i\rharpoon f)\otimes g_i.
    \end{equation}
    In consequence
    \begin{align*}
        (f\lharpoon a_j)(a) &= f(a_ja)\\
            &= \mu^*(f)(a_j\otimes a)\\
            &= \sum_{i=1}^n(a_i\rharpoon f)(a_j)g_i(a)\\
            &= g_j(a),
    \end{align*}
    i.e., $g_j=f\lharpoon a_j\in f\lharpoon A$. If $f$ vanishes in the left ideal $I$ of finite codimension, given $c\in I$ we have $g_j(c)=f(a_jc)=0$, i.e., $g_j$ vanishes on $I$. It follows that $g_j\in A^\circ$. The conclusion is now a direct consequence of~\eqref{eq:f-mu}.
\end{proof}

\begin{cor}\label{cor:algebra-to-coalgebra}
    Suppose that\/ $\kappa$ is a field. If\/ $A$ is\/ $\kappa$-algebra, then\/ $A^\circ$ is a\/ $\kappa$-coalgebra with\/ $\Delta=\mu^*$ and\/ $\varepsilon=\iota_A^*$.
\end{cor}

\begin{proof}
    The preceding corollary shows that $\mu*\colon A^*\to A^*\otimes_\kappa A^*$ induces, via restriction and coastriction, a map $\Delta\colon A^\circ\to A^\circ\otimes_\kappa A^\circ$ defined by
    $$
        \Delta(f) = \mu^*(f).
    $$
    We can also restrict $\iota_A^*$ to $A^\circ$ to get $\varepsilon\colon A^\circ\to\kappa$, which satisfies
    $$
        \varepsilon(f)=f\circ\iota_A.
    $$
    Put $\id=\id_{A^\circ}$. To verify that $A^\circ$ is a $\kappa$-coalgebra for $\Delta$ and $\varepsilon$ we have to prove the following equations
    $$
        (\id\otimes\Delta)\circ\Delta
            = (\Delta\otimes\id)\circ\Delta
        \quad\text{and}\quad
        (\varepsilon\otimes\id)\circ\Delta
            = (\id\otimes\varepsilon)\circ\Delta.
    $$
    For the first equation consider the diagrams
    \small
    $$
        \begin{tikzcd}
                &&[-0.5cm]&[-2cm]&&[-1.6cm]\\[-0.5cm]
                &&(A\otimes_kA\otimes_\kappa A)^*
                &&&(A\otimes_\kappa A)^*
                    \arrow[lll,"(\id_A\otimes\mu)^*"']\\[-0.2cm]
            A\otimes_\kappa A\otimes_\kappa A
                    \arrow[d,"\mu\otimes\id_A"']
                    \arrow[r,"\id_A\otimes\mu"]
                &A\otimes_\kappa A
                    \arrow[d,"\mu"]
                &&A^\circ\otimes_\kappa A^\circ\otimes_\kappa A^\circ
                    \arrow[lu]
                &A^\circ\otimes_\kappa A^\circ
                    \arrow[ru]
                    \arrow[l,"\id\otimes\Delta"']\\
            A\otimes_\kappa A
                    \arrow[r,"\mu"']
                &A
                &&A^\circ\otimes_\kappa A^\circ
                    \arrow[ld]
                    \arrow[u,"\Delta\otimes\id"]
                &A^\circ\arrow[rd]
                    \arrow[l,"\Delta"]
                    \arrow[u,"\Delta"']\\[-0.3cm]
                &&(A\otimes_\kappa A)^*
                    \arrow[uuu,"(\mu\otimes\id_A)^*"]
                &&&A^*
                    \arrow[uuu,"\mu^*"']
                    \arrow[lll,"\mu^*"]
        \end{tikzcd}
    $$
    \normalsize
    The one on the left commutes because $\mu$ is associative. In consequence, the outer square on the right, also commutes. Therefore, the inner one is commutative, which proves the coassociativity of $\Delta$.

    To verify the second equation, consider the dual of the diagram
    $$
        \begin{tikzcd}
            A\otimes_\kappa\kappa
                    \arrow[r,"\id\otimes_\kappa\iota_A"]
                    \arrow[d,"\mu_2"']
                &A\otimes_\kappa A\\
            A
                &\kappa\otimes_\kappa A
                    \arrow[u,"\iota_A\otimes_\kappa\id"']
                    \arrow[l,"\mu_1"]
        \end{tikzcd}
    $$
    After restriction and coastriction, we get
    $$
        \begin{tikzcd}[column sep=huge]
            (A\otimes_k\kappa)^*
                &(A\otimes_\kappa A)^*
                    \arrow[d,"(\iota_A\otimes\id_A)^*"]
                    \arrow[l,"(\id_A\otimes\iota_A)^*"']
                &[-0.5cm]A^\circ\otimes_\kappa\kappa
                &A^\circ\otimes A^\circ
                    \arrow[l,"\id\otimes\varepsilon"']
                    \arrow[d,"\varepsilon\otimes\id"]\\
            A^*\arrow[u,"\mu_2^*"]
                    \arrow[r,"\mu_1^*"']
                    \arrow[ru,"\mu^*"]&(\kappa\otimes_\kappa A)^*
                &A^\circ
                    \arrow[r,"1\otimes\id"']
                    \arrow[u,"\id\otimes1"]
                    \arrow[ru,"\Delta"]
                &\kappa\otimes_\kappa A^\circ
        \end{tikzcd}
    $$
    because $\mu_1^*=1\otimes\id$ and $\mu_2^*=\id\otimes1$. Indeed,
    \begin{align*}
        \mu_1^*(f)(k\otimes a) &= f\circ\mu_1(k\otimes a)\\
            &= f(ka)\\
            &= kf(a)\\
            &= (1\otimes f)(k\otimes a)\\
            &= (1\otimes\id)(f)(k\otimes a),
        \intertext{and}
        \mu_2^*(f)(a\otimes k) &= f\circ\mu_2(a\otimes k)\\
            &= f(ak)\\
            &= f(a)k\\
            &= (f\otimes1)(a\otimes k)\\
            &= (\id\otimes1)(f)(a\otimes k).
    \end{align*}
\end{proof}

\begin{rem}
    Following Remark~\ref{rem:coalgebra-convolution}, for every $\kappa$-module $K$ we can define the convolution in $\Hom_\kappa(A^\circ,K)$ to get
    \begin{align*}
        \alpha*\beta(f)
            &= \mu_K\circ(\alpha\otimes\beta)\circ\Delta(f)\\
            &= \mu_K\circ(\alpha\otimes\beta)(f\circ\mu_A).
    \end{align*}
    for $f\in A^\circ$.
    
    In the particular case were $f\colon A\to\kappa$ happens to be a morphism of $\kappa$-algebras, i.e., $f\circ\mu_A=f\otimes f$, we get
    \begin{align*}
        \alpha*\beta(f) &= \mu_K\circ(\alpha\otimes\beta)(f\circ f)\\
            &= (\alpha\circ f)(\beta\circ f),
    \end{align*}
    i.e.,
    $$
        \alpha*\beta(f)(a)=f(a)\alpha(1)\beta(1).
    $$
\end{rem}

\section{Quasitriangular Structures}

\begin{defn}
    Let $B$ be a $\kappa$-bialgebra and let $\units(B\otimes_\kappa B)$ denote the set of units in $B\otimes_\kappa B$. The pair $(B,\rho)$ is an \textsl{almost cocommutative $\kappa$-bialgebra} if  $\rho\in\units(B\otimes_\kappa B)$ and satisfies
    $$
        t\Delta_B(b) = \rho\Delta_B(b)\rho^{-1}
    $$
    for all $b\in B$.
\end{defn}

\begin{rem}
    If $B$ is cocommutative, $(B,1\otimes1)$ is almost cocommutative.
\end{rem}

\begin{ntn}
    Given\/ $\rho$ in the\/ $\kappa$-algebra\/ $B$,
    $$
        \rho = \sum_{i=1}^na_i\otimes b_i
    $$
    put
    \begin{align*}
        \rho^{12} &= \sum_{i=1}^na_i\otimes b_i\otimes1=\rho\otimes1\\
        \rho^{13} &= \sum_{i=1}^na_i\otimes1\otimes b_i\\
        \rho^{23} &= \sum_{i=1}^n1\otimes a_i\otimes b_i = 1\otimes\rho.
    \end{align*}
\end{ntn}

\begin{defn}
    The pair $(B,\rho)$ is a \textsl{quasitrianglular $\kappa$-bialgebra} if it is almost cocommutative and
    \begin{align*}
        (\Delta_B\otimes\id_B)(\rho) &= \rho^{13}\rho^{23}\\
        (\id_B\otimes\Delta_B)(\rho) &= \rho^{13}\rho^{12}.
    \end{align*}
    In that case, the element $\rho$ is called a \textsl{quasitriangular structure}.
\end{defn}

\begin{rem}
    If $B$ is cocommutative, $(B,1\otimes1)$ is quasitriangular.
\end{rem}

\begin{defn}
    A \textsl{morphism of quasitriangular $\kappa$-bialgebras} from $(B,\rho)$ to $(B',\rho')$ is a morphism of $\kappa$-bialgebras $\phi\colon B\to B'$ such that $\phi(\rho)=\rho'$.
\end{defn}

\begin{xmpl}
    Suppose that $\kappa$ is a field of characteristic different from~$2$. Let $C_2$ be the cyclic group of order $2$ generated by $\gamma$ and let $\kappa[C_2]$ be the group bialgebra. Then there are two non-equivalent quasitriangular structures on $\kappa[C_2]$, namely $1\otimes1$ and
    $$
        \rho=\tfrac12(1\otimes(1+\gamma)+\gamma\otimes(1-\gamma)).
    $$
    We first need to verify that $\rho$ is a unit. But
    \begin{align*}
        \rho^2
            &= \tfrac14(1\otimes(1+\gamma)^2+1\otimes(1-\gamma)^2)
                +2(\cancel{(1\otimes(1+\gamma))(\gamma\otimes(1-\gamma)})
                \\
            &= \tfrac14(1\otimes(2+2\gamma^2))\\
            &= 1\otimes1.
    \end{align*}
    Recall from Exercise~\ref{exr:ring-group-hopf-algebra} that
    $$
        \Delta(a+b\gamma) = a(1\otimes1) + b(\gamma\otimes\gamma).
    $$
    
    Since $\kappa[C_2]$ is commutative and cocommutative, $(\kappa[C_2],\rho)$ is almost cocommutative. Moreover,
    \begin{align*}
        \rho^{12} &= \tfrac12(1\otimes(1+\gamma)\otimes1
            +\gamma\otimes(1-\gamma)\otimes1)\\
        \rho^{13} &= \tfrac12(1\otimes1\otimes(1+\gamma)
            +\gamma\otimes1\otimes(1-\gamma))\\
        \rho^{23} &= \tfrac12(1\otimes1\otimes(1+\gamma)
            +1\otimes\gamma\otimes(1-\gamma)).
    \end{align*}
    Observe that
    \begin{align}
        (1+\gamma)\otimes(1+\gamma)+(1-\gamma)\otimes(1-\gamma)
            &=
            1\otimes1+1\otimes\gamma+\gamma\otimes1+\gamma\otimes\gamma
                \nonumber\\
            &\quad+ 1\otimes1-1\otimes\gamma-
            \gamma\otimes1+\gamma\otimes\gamma)\nonumber\\
            &= 2(1\otimes1+\gamma\otimes\gamma)\label{eq:11+gg}.\\
        (1-\gamma)\otimes(1+\gamma)+(1+\gamma)\otimes(1-\gamma)
            &= 1\otimes1+1\otimes\gamma-\gamma\otimes1
                -\gamma\otimes\gamma\nonumber\\
            &\quad+ 1\otimes1-1\otimes\gamma+\gamma\otimes1
                -\gamma\otimes\gamma\nonumber\\
            &= 2(1\otimes1-\gamma\otimes\gamma)\label{eq:11-gg}.
    \end{align}
    Therefore,
    \begin{align*}
        \rho^{13}\rho^{23} &=
            \tfrac14(1\otimes1\otimes(2+2\gamma)
            +\gamma\otimes\gamma\otimes(2-2\gamma))\\
            &=\tfrac12(1\otimes1\otimes(1+\gamma)
                +\gamma\otimes\gamma\otimes(1-\gamma)).\\
        \rho^{13}\rho^{12} &=
            \tfrac14(1\otimes(1+\gamma)\otimes(1+\gamma)
            + 1\otimes(1-\gamma)\otimes(1-\gamma)\\
            &\quad +\gamma\otimes(1-\gamma)\otimes(1+\gamma)
            +\gamma\otimes(1+\gamma)\otimes(1-\gamma)\\
            &= \tfrac12(1\otimes1\otimes1+1\otimes\gamma\otimes\gamma
            +\gamma\otimes1\otimes1-\gamma\otimes\gamma\otimes\gamma)
                &&;\ \eqref{eq:11+gg}\text{ \& }\eqref{eq:11-gg}.
    \end{align*}
    On the other hand,
    \begin{align*}
        \Delta\otimes\id(\rho) &=
            \tfrac12(1\otimes1\otimes(1+\gamma)
            +\gamma\otimes\gamma\otimes(1-\gamma))
            = \rho^{13}\rho^{23}.\\
        \id\otimes\Delta(\rho) &=
            \tfrac12(1\otimes(1\otimes1+\gamma\otimes\gamma)
            +\gamma\otimes(1\otimes1-\gamma\otimes\gamma))
            = \rho^{13}\rho^{12}.
    \end{align*}
\end{xmpl}

\begin{thm}\label{thm:drinfeld} {\rm[Drinfeld]}.
    Let $(B,\rho)$ be a quasitriangular $\kappa$-bialgebra. Then
    \begin{equation}\label{eq:qybe}
        \rho^{12}\rho^{13}\rho^{23}= \rho^{23}\rho^{13}\rho^{12}.
    \end{equation}
\end{thm}

\begin{proof}
    By definition we must prove that
    $$
        \rho^{12}(\Delta\otimes\id)(\rho)
            = \rho^{23}\rho^{13}\rho^{12}.
    $$
    Put
    $$
        \rho = \sum_{i=1}^na_i\otimes b_i.
    $$
    Firstly observe that
    \begin{align}
        (t\otimes\id)(\rho^{13}\rho^{23})&= \sum_{i=1}^n\sum_{j=1}^n
                (t\otimes\id)
                ((a_i\otimes1\otimes b_i)(1\otimes a_j\otimes b_j))
                \nonumber\\
            &= \rho^{23}\rho^{13}\label{eq:trr-rr}.
    \end{align}
    Then,
    \begin{align*}
        \rho^{12}(\Delta\otimes\id)(\rho)
            &= (\rho\otimes1)\sum_{i=1}^n\Delta(a_i)\otimes b_i\\
            &= \sum_{i=1}^n\rho\Delta(a_i)\otimes b_i\\
            &= \sum_{i=1}^nt\circ\Delta(a_i)\rho\otimes b_i\\
            %&= \sum_{i=1}^n
            %    (t\circ\Delta(a_i)\otimes b_i)(\rho\otimes1)\\
            &= \sum_{i=1}^n
                ((t\circ\Delta\otimes\id)(a_i\otimes b_i))(\rho\otimes1)\\
            &= (t\circ\Delta\otimes\id)(\rho)(\rho\otimes1)\\
            &= (t\otimes\id)\circ(\Delta\circ\id)(\rho)\rho^{12}\\
            &= (t\otimes\id)(\rho^{13}\rho^{23})\rho^{12}\\
            &= \rho^{23}\rho^{13}\rho^{12}
                &&;\ \eqref{eq:trr-rr}.
    \end{align*}
\end{proof}

\begin{rem}
    Equation \eqref{eq:qybe} is known as the \textsl{quantum Yang-Baxter equation}.
\end{rem}

