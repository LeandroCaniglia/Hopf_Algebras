\usepackage{amsthm}
%\usepackage{graphicx}
\usepackage{parskip}
\usepackage{amsmath}
\usepackage[shortlabels]{enumitem}
%\usepackage{supertabular}
\usepackage{mathtools}
\usepackage{amssymb}
\usepackage{hyperref}
\usepackage{tikz-cd}
\usepackage{chngcntr}
%\usepackage{caption}
\usepackage{tikz}
\usetikzlibrary{intersections}
\usetikzlibrary{patterns}
\usetikzlibrary{matrix,arrows,decorations.pathmorphing}
\usetikzlibrary{positioning,decorations.markings,quotes}
%\usepackage{subdepth}
%\usepackage[mathletters]{ucs}
%\usepackage[utf8x]{inputenc}
%\usepackage{newunicodechar}
\usepackage{cancel}
\usepackage{enumitem}
\tikzset{close/.style={outer sep=-2pt}}
\usetikzlibrary{positioning, quotes}
%\usetikzlibrary[fit]
\usepackage{needspace}
%\usepackage{mathrsfs}
\usepackage[outline]{contour}
\usepackage[T1]{fontenc}
\contourlength{0.1pt}
\contournumber{10}
%\usepackage [cal=dutchcal]{mathalfa}
\usepackage{blindtext}
\usepackage[square,sort,comma]{natbib}
%\usepackage{pgfplots}
%\usepackage{mathrsfs}
%\usepackage[bb=libus]{mathalpha}


\hypersetup{
    colorlinks=true,
    citecolor=black,  % Set citation color to black
    linkcolor=darkgray,
    urlcolor=darkgray
    }
%\usepackage[utf8]{inputenc}
 
%{\fontfamily{cmss}\selectfont
%This text uses a different font typeface
%}

%NEW COMMANDS
\newcommand{\N}{\mathbb{N}}
\newcommand{\R}{\mathbb{R}}
\newcommand{\C}{\mathbb{C}}
\newcommand{\Z}{\mathbb{Z}}
\newcommand{\Q}{\mathbb{Q}}
\newcommand{\op}[1]{\operatorname{#1}}
\newcommand{\Rn}{\R^n}
\newcommand{\Rm}{\R^m}
\newcommand{\Rk}{\R^k}
\newcommand{\Rd}{\R^d}
\newcommand{\Rnxn}{\R^{n\times n}}
\newcommand{\Rmxn}{\R^{m\times n}}
\newcommand{\Rkxn}{\R^{k\times n}}
\newcommand{\Rnxm}{\R^{n\times m}}
\newcommand{\im}[1]{\text{\rm im}#1}
\newcommand{\id}{\text{\rm id}}
\newcommand{\dom}{\text{\rm dom}}
\newcommand{\overbar}[1]{\,\overline{\!{#1}}}
\newcommand{\seq}[2][n]{(#2)_{#1\in\N}}
\newcommand{\set}[1]{\{#1\}}
\newcommand{\blank}{\vbox{\medskip\medskip}}
\newtheorem{thm}{\blank Theorem}[section]
\newtheorem{ntn}[thm]{\blank Notation}
\newtheorem{ntns}[thm]{\blank Notations}
\newtheorem{lem}[thm]{\blank Lemma}
\newtheorem{cor}[thm]{\blank Corollary}
\newtheorem{prop}[thm]{\blank Proposition}
\newtheorem{probl}[thm]{\blank Problem}
\newtheorem{exr}[thm]{\blank Exercise}
\newtheorem{ill}[thm]{Illustration}
\theoremstyle{definition}
\newtheorem{defn}[thm]{Definition}
\newtheorem{defns}[thm]{Definitions}

\theoremstyle{definition}
    \newtheorem{rem}[thm]{\blank Remark}
    \newtheorem{rems}[thm]{\blank Remarks}
    \newenvironment{solution}[1][]
      {\begin{proof}[Solution#1]${}$}
      {\end{proof}}
    \newtheorem{xmpl}[thm]{Example}
    \newtheorem{xmpls}[thm]{Examples}
\theoremstyle{plain}

\counterwithin*{equation}{section}
\counterwithin*{equation}{subsection}
\newcommand\Ccancel[2][black]{\renewcommand\CancelColor{\color{#1}}\cancel{#2}}
\newcommand{\mbf}[1]{\textrm{\boldmath$#1$}}

% widebar (https://tex.stackexchange.com/a/60253/102342)
    \makeatletter
    \let\save@mathaccent\mathaccent
    \newcommand*\if@single[3]{%
      \setbox0\hbox{${\mathaccent"0362{#1}}^H$}%
      \setbox2\hbox{${\mathaccent"0362{\kern0pt#1}}^H$}%
      \ifdim\ht0=\ht2 #3\else #2\fi
      }
    %The bar will be moved to the right by a half of \macc@kerna, which is computed by amsmath:
    \newcommand*\rel@kern[1]{\kern#1\dimexpr\macc@kerna}
    %If there's a superscript following the bar, then no negative kern may follow the bar;
    %an additional {} makes sure that the superscript is high enough in this case:
    \newcommand*\widebar[1]{\@ifnextchar^{{\wide@bar{#1}{0}}}{\wide@bar{#1}{1}}}
    %Use a separate algorithm for single symbols:
    \newcommand*\wide@bar[2]{\if@single{#1}{\wide@bar@{#1}{#2}{1}}{\wide@bar@{#1}{#2}{2}}}
    \newcommand*\wide@bar@[3]{%
      \begingroup
      \def\mathaccent##1##2{%
    %Enable nesting of accents:
        \let\mathaccent\save@mathaccent
    %If there's more than a single symbol, use the first character instead (see below):
        \if#32 \let\macc@nucleus\first@char \fi
    %Determine the italic correction:
        \setbox\z@\hbox{$\macc@style{\macc@nucleus}_{}$}%
        \setbox\tw@\hbox{$\macc@style{\macc@nucleus}{}_{}$}%
        \dimen@\wd\tw@
        \advance\dimen@-\wd\z@
    %Now \dimen@ is the italic correction of the symbol.
        \divide\dimen@ 3
        \@tempdima\wd\tw@
        \advance\@tempdima-\scriptspace
    %Now \@tempdima is the width of the symbol.
        \divide\@tempdima 10
        \advance\dimen@-\@tempdima
    %Now \dimen@ = (italic correction / 3) - (Breite / 10)
        \ifdim\dimen@>\z@ \dimen@0pt\fi
    %The bar will be shortened in the case \dimen@<0 !
        \rel@kern{0.6}\kern-\dimen@
        \if#31
          \overline{\rel@kern{-0.6}\kern\dimen@\macc@nucleus\rel@kern{0.4}\kern\dimen@}%
          \advance\dimen@0.4\dimexpr\macc@kerna
    %Place the combined final kern (-\dimen@) if it is >0 or if a superscript follows:
          \let\final@kern#2%
          \ifdim\dimen@<\z@ \let\final@kern1\fi
          \if\final@kern1 \kern-\dimen@\fi
        \else
          \overline{\rel@kern{-0.6}\kern\dimen@#1}%
        \fi
      }%
      \macc@depth\@ne
      \let\math@bgroup\@empty \let\math@egroup\macc@set@skewchar
      \mathsurround\z@ \frozen@everymath{\mathgroup\macc@group\relax}%
      \macc@set@skewchar\relax
      \let\mathaccentV\macc@nested@a
    %The following initialises \macc@kerna and calls \mathaccent:
      \if#31
        \macc@nested@a\relax111{#1}%
      \else
    %If the argument consists of more than one symbol, and if the first token is
    %a letter, use that letter for the computations:
        \def\gobble@till@marker##1\endmarker{}%
        \futurelet\first@char\gobble@till@marker#1\endmarker
        \ifcat\noexpand\first@char A\else
          \def\first@char{}%
        \fi
        \macc@nested@a\relax111{\first@char}%
      \fi
      \endgroup
    }

%groups
\newcommand{\gen}[1]{\mbf{(}#1\mbf{)}}
\newcommand{\ord}[1]{{\rm ord}#1}
\newcommand{\Sym}[1]{{\rm Sym}#1}
\newcommand{\Alt}[1]{{\rm Alt}#1}
\newcommand{\Aut}[1]{{\rm Aut}#1}
\newcommand{\Hom}{{\rm Hom}}
\newcommand{\End}[1]{{\rm End}#1}
\newcommand{\Inn}[1]{{\rm Inn}#1}
\newcommand{\Syl}[1]{{\rm Syl}#1}
\newcommand{\Hall}[1]{{\rm Hall}#1}
\newcommand{\Core}{{\rm Core}}
\newcommand{\normal}{\lhd}
\newcommand{\snormal}{\mathrel{\triangleleft\triangleleft}}
\newcommand{\comm}{\mathrel\leftrightarrow}
\newcommand{\cnjcls}[1]{\left[#1\right]}
\newcommand{\spec}[1]{\text{\rm spec}#1}
\DeclareMathOperator{\lcm}{lcm}
\DeclareMathOperator{\ad}{ad}
\DeclareMathOperator{\ct}{ct}
\newcommand{\subgroup}{\leqslant}
\DeclareMathOperator{\CD}{{\cal CD}}
\DeclareMathOperator{\Nm}{{\cal N}_m}
\newcommand{\Soc}[1]{{\rm Soc}#1}
\newcommand{\sd}{{\rm sd}}
\newcommand{\ch}{\mathrel{\textcolor{black!65}\blacktriangleleft}}
\newcommand{\Fp}[1][n]{\mathbb{F}_{p^{#1}}}
\newcommand{\F}{\mathbb{F}}
\newcommand{\nnormal}{\mathrel{\ooalign{\hss$/$\hss\cr$\normal$}}}
\DeclareMathOperator{\Hol}{Hol}

\DeclareMathSymbol{\skw}{\mathbin}{letters}{"3F}

\newcommand{\cat}[1]{\mathsf{#1}}

% Algebras

\DeclareMathOperator{\Spec}{Spec}
\DeclareMathOperator{\Max}{Max}
\DeclareMathOperator{\Ann}{Ann}
\DeclareMathOperator{\Nil}{Nil}
\DeclareMathOperator{\Frac}{Frac}
\newcommand{\foraall}{\forall{\vphantom{|}}^*} %for almost all
\makeatletter
\newcommand{\aew}[1][\@empty]{%
  \ifx#1\@empty
    {\text{a.e.}}%
  \else
    ,{\ \text{a.e.\ in } #1}%
  \fi
}
\makeatother
\DeclareMathOperator{\cl}{cl}
\newcommand{\sheaf}[1]{\mathcal{#1}}
\newcommand{\qsheaf}[1]{\mathcal Q_{\!#1}}
\newcommand{\osheaf}[1]{\mathcal O_{#1}}

\newcommand\scalemath[2]{\scalebox{#1}{\mbox{\ensuremath{\displaystyle #2}}}}

\newcommand{\ev}{\text{\rm ev}}
\newcommand{\xto}[1]{\xrightarrow{#1}}

\newcommand{\curly}{\mathrel{\leadsto}}

\newcommand{\yoneda}{\mbf{\mathit h}}
\newcommand{\yon}{\mathtt h}
\newcommand{\units}{\mathbf U}
\newcommand{\rharpoon}{\rightharpoonup}
\newcommand{\lharpoon}{\leftharpoonup}

