\chapter{Monads}

\section{Monads and Comonads}

Let $\cat C$, $\cat D$ and $\cat E$ be categories. Let $G\colon\cat E\to\cat C$ and $F_1,F_2\colon \cat C\to\cat D$ be functors and $\eta\colon F_1\to F_2$ a natural transformation. By composition we get
\begin{align*}
    \eta G\colon F_1\circ G&\to F_2\circ G\\
        (\eta G)_Z = \eta_{G(Z)}\colon F_1\circ G(Z)&\to F_2\circ G(Z)
            %&&(Z\in\op{Obj}(\cat E))
\end{align*}
which is a natural transformation from $F_1\circ G$ to $F_2\circ G$. Indeed, the diagram
$$
    \begin{tikzcd}
        F_1\circ G(Z)
            \arrow[r,"\eta_{G(Z)}"]
            \arrow[d,"F_1\circ G(\psi)"']
        &F_2\circ G(Z)
            \arrow[d,"F_2\circ G(\psi)"]\\
        F_1\circ G(Z')
            \arrow[r,"\eta_{G(Z')}"']
        &F_2\circ G(Z')
    \end{tikzcd}
$$
commutes for every $\psi\colon Z\to Z'$ because it expresses the naturality of $\eta$ at the morphism $G(\psi)\colon G(Z)\to G(Z')$.

Similarly, in the case where $G\colon\cat D\to\cat E$, the composition
\begin{align*}
    G\eta\colon G\circ F_1&\to G\circ F_2\\
    (G\eta)_X=G(\eta_X)\colon G\circ F_1(X)&\to G\circ F_2(X)
        %&&(X\in\op{Obj}(\cat C))
\end{align*}
defines a natural transformation because, for every $\phi\colon X\to X'$, the diagram
$$
    \begin{tikzcd}
        G\circ F_1(X)
                \arrow[r,"G(\eta_X)"]
                \arrow[d,"G\circ F_1(\phi)"']
            &G\circ F_2(X)
                \arrow[d,"G\circ F_2(\phi)"]\\
        G\circ F_1(X')
            \arrow[r,"G(\eta_{X'})"']
        &G\circ F_2(X')
    \end{tikzcd}
$$
results from applying $G$ to the diagram that expresses the naturality of~$\eta$ at~$\phi$.

\begin{defn}
    Let $\cat C$ be a category and let $I_{\cat C}\colon\cat C\to\cat C$ denote the identity functor. A function $T\colon\cat C\to\cat C$ is a \textsl{monad} when it is equipped with two natural transformations: a \textsl{unit} $\eta\colon I_{\cat C}\to T$ and a \textsl{multiplication} $\mu\colon T^2\to T$, where $T^2=T\circ T$, that make commutative the following diagrams
    $$
        \begin{tikzcd}
            T
                    \arrow[r,"\eta T"]
                    \arrow[d,"T\eta"']
                    \arrow[rd,equal]
                &T^2
                    \arrow[d,"\mu"]
                &T^3
                    \arrow[d,"\mu T"']
                    \arrow[r,"T\mu"]
                &T^2
                    \arrow[d,"\mu"]\\
            T^2
                    \arrow[r,"\mu"']
                &T
                &T^2
                    \arrow[r,"\mu"']
                &\hphantom.T.
        \end{tikzcd}
    $$
\end{defn}

\begin{xmpl}
    Let $\mbf2$ be the $\cat{Set}$ endofunctor defined by
    \begin{align*}
        X&\mapsto 2^X\\
        X\xto\phi Y&\mapsto2^X\xto{2^\phi}2^Y\\
        \hphantom{X\xto\phi Y}&\hphantom{\mapsto\;\;\;}A\;\mapsto\phi(A).
    \end{align*}
    Put $\eta\colon I\to\mbf2$ as the inclusion $\eta_X\colon X\hookrightarrow 2^X$ that maps $x$ to $\set x$, and $\mu\colon\mbf2^{\mbf 2}\to\mbf2$ as the union $\mu_X\colon 2^{2^X}\to2^X$ that maps $\mathcal C$ to $\bigcup\mathcal C=\bigcup_{A\in\mathcal C}A$. For $A\in 2^X$ we have
    \begin{align*}
        (\mu\circ\eta\mbf2)_X(A) & = \mu_X\circ\eta_{2^X}(A)\\
            &= \mu_X(\set A)\\
            &= \bigcup\set A\\
            &= A.
        \intertext{Similarly,}
        (\mu\circ\mbf2\eta)_X(A) &= \mu_X\circ2^{\eta_X}(A)\\
            &= \mu_X(\eta_X(A))\\
            &= \mu_X(\set{\eta_X(a)\mid a\in A}\\
            &= \bigcup\set{\set a\mid a\in A}\\
            &= A,
    \end{align*}
    which shows that the first diagram commutes.
    
    For the commutativity of the second diagram take $\mathcal A\in2^{2^{2^X}}$. Then
    \begin{align*}
        (\mu\circ\mbf2\mu)_X(\mathcal A)
            &= \mu_X\circ 2^{\mu_X}(\mathcal A)\\
            &= \mu_X(\set{\mu_X(A)\mid A\in\mathcal A})\\
            &= \bigcup\Big\{\bigcup_{B\in A}B\mid A\in\mathcal A\Big\}\\
            &= \bigcup_{A\in\mathcal A}\bigcup_{B\in A}B.
        \intertext{and}
        (\mu\circ\mu\mbf2)_X(\mathcal A)
            &= \mu_X\circ\mu_{2^X}(\mathcal A)\\
            &= \mu_X\Big(\bigcup_{A\in\mathcal A}A\Big)\\
            &= \bigcup\Big\{B\mid B\in\bigcup_{A\in\mathcal A}A\Big\}
    \end{align*}
    The equality is now easy to verify by double inclusion.
\end{xmpl}